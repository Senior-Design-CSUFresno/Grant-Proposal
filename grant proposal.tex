%Written by: Aaron Stillmaker
%September 11, 2018
%ECE 186A - Senior Design
%
%This is a template you can use for your Project Descriptions, though you will need to change a good portion of it for your own needs.

\documentclass{IEEEtran}					%Everything is default, which is journal, 10pt font, and final draft
\usepackage{dtk-logos}						%This let me do the cool BibTeX logo, you shouldn't need this line. Add additional packages here:
\usepackage{graphics}
\usepackage{caption} 
\usepackage{natbib}
\captionsetup[table]{skip=10pt}
\title{\vspace{2in}The Smart Eraser}	%the \vspace is moving the title down from the top of the page
\author{ECE 186A - Senior Design I \\ 			%I am jamming all of the stuff I want on the title page into the Author spot, I know, 
	%this isn't elegant.
	Fall 2018 - Dr. Stillmaker \\ 				%Note the \\ line returns to make sure to put each one on a new line
	\vspace{12pt} 								%I put in some space to get the due date listed a little lower
	\textbf{Date:} Friday, September 28, 2018  \\ 
	\vspace{2in}								%You may need to mess with this space, this space is above the signature lines.
	\vspace{6pt}
	\textbf{Project Manager:} Heather Libecki			%Uncomment this section and fill in your team names and your technical advisor's name.
	\vspace{12pt}
	\underline{\hspace{3in}}\\					%You are welcome to have more than one technical advisor if you wish..
	\textbf{Team Member:} Chris Quesada
	\vspace{12pt}
	\underline{\hspace{3in}}\\
	\textbf{Team Member:} Juan Colin
	\vspace{12pt}
	\underline{\hspace{3in}}\\
	\textbf{Technical Advisor:} Dr. Kulhandjian
	\underline{\hspace{3in}} \\
	\vspace{12in}}								%You will need to mess with this space, this space is after the signature lines.  The idea
%is to make the title page by itself.  I know, this isn't elegant either, but IEEE %formatting doesn't make title pages

\begin{document}
	\thispagestyle{empty}						%This removes the page number from the title page.
	
	\maketitle									%This generates the title from the information given above.
	
	
	\section{Smart Eraser Description}
	\setlength{\parindent}{5ex}
	Our project, the Smart Eraser, is an automatic whiteboard eraser. The main deliverable of this project will be an eraser which can move left-to-right on a track, and up-and-down on a linear motion system attached to the track. This eraser will be able to detect where markings are on a whiteboard through the use of a camera and an image-processing program. The camera will send the image of the whiteboard to a microcontroller which will process the image, detect where the markings are, and convert their locations to a coordinate system that the mechanical aspects of the eraser will be able to read. The eraser will then find the quickest route to erase all of the markings before returning to its stand-by position. Finally, the eraser will be able to detect the presence of a person through the use of the camera and motion-detection technology. This will check if there is a person moving in front of the whiteboard, and if there is, the result would be an immediate termination of the process the Smart Eraser was carrying out; this is to ensure the safety of those around the Smart Eraser while it is operating.
	
	\section{Impact of the Smart Eraser}
	\setlength{\parindent}{5ex}
	The Smart Eraser aims to assist teachers and students who want to focus more on the material being taught and less on the clean up afterward. Have you ever been in a classroom and your teacher fills the whole whiteboard with notes and various examples, only to use precious class time to erase it everytime they need more space to write? With the Smart Eraser, teachers could leave that tedious task to the machines. In between examples, a teacher would be able to clear the board with the press of a button, and then continue teaching without worry.
	\setlength{\parindent}{5ex}
	Our project would require a railing system to connect to a whiteboard, as well as a place to mount the camera to be used for image processing. With California State University, Fresno in mind specifically, each classroom already has a projector mounted on the ceiling which points toward the whiteboard at the front of the classroom. The Smart Eraser would utilize this mounted space to place the camera, and wall mounts would be used to put the track system on the board itself.
	\setlength{\parindent}{5ex}
	At the beginning of this project, research was done to see if there were any similar models or products that already existed which accomplish what the Smart Eraser would do. Various “automatic whiteboard eraser” projects were found, but they were focused more on the mechanical movement of the eraser, and less on the brains behind one that could detect where it needs to go in a minimal amount of time. Therefore, the Smart Eraser will be unique in its capabilities to detect not only where it needs to go on the whiteboard, but also how to get those markings erased in the shortest time possible.
	
	\section{Budget Estimate}
	\setlength{\parindent}{5ex}
		\begin{table} [h!]	
		\centering
		\resizebox{8cm}{!}{
			\begin{tabular}{|c|c|}
				\hline
				\textbf{Component} & \textbf{Est. Price} \\
				\hline
				DE1-Soc FPGA Development Board - Terasic & \$175 \\
				\hline
				CNC stepper motor driver - STEPPERONLINE & \$67.90 (\$33.95) \\
				\hline
				Carriage with Stainless Steel Balls; for use with 115RC Linear Tack - Accuride & \$67.16 (\$33.58 x2) \\ 
				\hline 		
				115RC 47in Linear Motion Aluminum Track Systems -Accuride & \$62.68 (\$31.34 x2) \\
				\hline
				Nema 23 CNC 2.8A Stepper Motor - STEPPERONLINE & \$52.00 (\$26.00) \\
				\hline
				1080p POE Camera - sv3c & \$38.99 \\
				\hline
				Dry Erase Board (prototype) 36" x 24" - VIZ-PRO & \$31.90 \\
				\hline
				Stepper Motor Encasing - D.Y Engineering & \$25.98 (\$12.99 X2) \\
				\hline
				5 Meter GT2 timing Belt (6mm width) - Mercury & \$17.98 (\$8.99 X2) \\
				\hline
				6.35mm GT2 40 Teeth Pulley Flange - uxcell & \$14.38 (\$7.19 X2) \\
				\hline
				Nema 23 Stepper Motor Steel Mount Bracket w/ Screws - HobbyUnlimited & \$10.99 \\
				\hline
				Dry Erase Whiteboard Block Eraser - Expo & \$8.90 \\
				\hline
				PCB for H Bridge (for Stepper Driver) and Stepper Motors, possible LCD screen & \ Unknown \\
				\hline
				Various Wires and Connection Cables & \ Unknown \\
				\hline 
				\textbf{Total Rough Budget} & \ \textbf{\$573.86} \\
				\hline 
		\end{tabular} }
		\caption{Estimated costs of components for project}
		\label{table:1}
	\end{table}	
	
	\section{Project Schedule}
	\setlength{\parindent}{5ex}
	
	\section{Lab and Other Resources Needed}
	\setlength{\parindent}{5ex}	
	
	\section{Project Team Bios}
	\setlength{\parindent}{5ex}
	
	\section{Smart Eraser Description}
	\setlength{\parindent}{5ex}
	
	\section{Economic Analysis}
	\setlength{\parindent}{5ex}
	
	\section{Advisor Recommendation}
	\setlength{\parindent}{5ex}

\end{document}
